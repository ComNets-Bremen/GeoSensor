\setlang{german}
\thispagestyle{empty}
\def\abstractname{Kurzfassung}

\begin{abstract}

  In dieser Abschlussarbeit wird der Entwurf eines Systems beschrieben, dass der georeferenzierten, also mit geografischen Koordinaten verbundenen, Messdatenerfassung dient. Als technologische Grundlagen werden hierbei die Arduino-Plattform als eingebettetes System, an dem Sensoren angeschlossen werden, und die Android-Plattform zur Darstellung, Speicherung und zum Export der Daten verwendet. F"ur die Arduino-Plattform wurde ein Software entwickelt, die eine Integration verschiedener Sensoren entsprechend des Bedarfes der aktuellen Anwendung erm"oglicht. Zur Kommunikation mit dem Android-Ger"at wird Bluetooth verwendet, wof"ur die Arduino-Hardware mit einem entsprechenden Modul verbunden werden muss. Auf dem Android-Ger\"at wird eine Applikation installiert, die alle vorgenannten Funktionen inklusive einer Visualisierung der Messdaten auf einer Stra\ss{}enkarte beeinhaltet. Die Daten k\"onnen aus der App in eine .csv-Datei exportiert werden, die mit anderen Apps ge\"offnet oder auf andere Ger\"ate \"ubertragen werden kann.

\end{abstract}

\setlang{USenglish}