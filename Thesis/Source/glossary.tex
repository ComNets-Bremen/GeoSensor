\label{cha:glossar}
\begin{deflist}{Interoperabilit�t}
\item [Android] Mobile operating system based on Linux developed by the Open Handset Alliance under the patronage of Google
\item [Arduino] Integrated open-source hard- and software for microcontroller projects
\item [AT-Commands] Many modems and other communication circuit components can be controlled using the Hayes command set and accompanying proprietary extensions. The commands are transmitted via a serial connection (often TTL) and normally start with AT and are thus called AT-commands.
\item [Beidou] The BeiDou Navigation Satellite System is a partly operational GNSS operated by the China National Space Administration
\item [Bluetooth] Wireless technology for personal area networks using the 2.4 GHz ISM-band
\item [Gallileo] Gallileo is a currently partially active civil GNSS supervised by the European Global Navigation Satellite Systems Agency (i.e. the European Union)
\item [Global Positioning System ] The Global Positioning System (GPS) is a military GNSS operated by the Air Force Space Command of the United States of America
\item [GLONASS] GLONASS is a military GNSS operated by the Russian Aerospace Defence Forces
\item [Location Provider] A location provider in Android is an API allowing an Android Application to receive location updated provided by some means of geolocalisation in android. Android devices commonly offer three location providers, one based on GPS and AGPS, one based on cellular network trilateration and avaliable WiFi networks and the third delivering only information when another app uses one of the other providers. Apart from that the Google Play services offer a fused location provider that combines the aforementioned location providers.
\item [NMEA sentence]
The National Marine Electronics Association (of the USA) defined the standard NMEA 0183 for a communication system used by various types of marine electronics. The standard describes a data format compromised of so-called sentences the can be transmitted over serial communication systems. The standard proposes the use of RS-422, however GPS modules offered for the Arduino platform will normally use TTL serial communication.
\end{deflist}

